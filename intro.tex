The analysis of data streams comes along with important questions: what kind of
data is it? What important information is contained in it? How does the stream
evolve? The key question for this project among those is the latter, i.e. dealing with
the evolution of the stream, because prior to the development of the CluStream \cite{clustreamOrig}
method there was not an easy to answer that question as it was one of the first to
tackle this issue.

Clustering is one of the main tasks in data mining, also often referred as an
exploratory subtask of it. As the name implies, the objective is to find clusters, i.e.,
collections of objects that share common properties. One can also relate this task
to unsupervised machine learning, which intends to classify data when it lacks of
labels, i.e., when the data instance does not indicate to which category it belongs.
The CluStream method was developed in 2003 \cite{clustreamOrig} and its main purpose is to pro-
vide more information than previously developed algorithms for data stream clus-
tering by that time. It provides a solution for handling streams of data indepen-
dently from the one that finds the final clusters. It consists of two phases (passes)
instead of one; the first one deals with the incoming data and stores relevant in-
formation over time and the second one is in charge of the clustering using the
previously generated information. In other words,

% For each batch of data, statistically relevant �summaries� of the data are created and stored at a defined pace. This storing pace follows a specific storage scheme such that the disk space requirement reduces drastically; this is necessary as in most cases for data streams one does not want to store everything
% that arrives, one reason being the big data requires large and expensive computational resources (processing power and storage).
% On user demand, the stored �summaries can be used for the end clustering
% task as they include all necessary information to achieve accurate results.
% Additionally, as these summaries are stored over time, a user defined time
% horizon/window can be chosen in order to analyze the data in different time
% periods, giving the possibility of a better understanding of the evolution of
% the data.
