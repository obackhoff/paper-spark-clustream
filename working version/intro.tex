The analysis of data streams comes with important questions: what kind of
data is it? What important information is contained in the stream? How does the stream
evolve? 
The later comprises the key question for our work, i.e., dealing with the evolution of the stream. 
We focus on the clustering task, i.e., how can we maintain a valid clustering structure in a volatile and evolving data stream 
environment.
 
Clustering is one of the main tasks in data mining, also referred as an exploratory task as it allows us to get to know our data. 
The goal is to  find groups of objects that share similar characteristics. 
% One can also relate this task to unsupervised machine learning, which intends to classify data when it lacks of
% labels, i.e., when the data instance does not indicate to which category it belongs.
One of the most popular algorithms for stream clustering is CluStream~\cite{clustreamOrig}, proposed in 2003, which provides more information than previously 
developed algorithms for data stream clustering by that time. This is achieved by separating the clustering process into two parts, an offline part that summarizes the stream in an online setting through appropriate summaries, the so-called microclusters, and an offline part that derives the final clusters upon these summaries. The first part is online, whereas the second is offline.

% In other words, i) For each batch of data, statistically relevant �summaries� of the data are cre-
% ated and stored at a defined pace. This storing pace follows a specific storage
% scheme such that the disk space requirement reduces drastically; this is nec-
% essary as in most cases for data streams one does not want to store everything
% that arrives, one reason being the big data requires large and expensive com-
% putational resources (processing power and storage).
% ii) On user demand, the stored �summaries can be used for the end clustering
% task as they include all necessary information to achieve accurate results.
% Additionally, as these summaries are stored over time, a user defined time
% horizon/window can be chosen in order to analyze the data in different time
% periods, giving the possibility of a better understanding of the evolution of
% the data.
Recently, distributed computing frameworks are employed to deal with big data challenges and many algorithms have been adapted to these settings.
In this work, we adapt CluStream to the Apache Spark framework in order to obtain a distributed stream clustering algorithm that can deal with clustering over big data streams.
As CluStream was not designed for a distributed setting, the adaptation is not straightforward and involves re-designing core parts of the algorithm.

